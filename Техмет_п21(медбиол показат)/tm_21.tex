% !TeX spellcheck = ru_RU-Russian
\documentclass[11pt]{article}
\usepackage{ucs} 
\usepackage[utf8x]{inputenc} % Включаем поддержку UTF8  
\usepackage[russian]{babel}  % Включаем пакет для поддержки русского языка 
\usepackage {mathtext}
\usepackage{mathrsfs, amsmath, amssymb}
\usepackage{graphicx}
\usepackage{listings}
\usepackage{hyperref}
\usepackage{revsymb}
\usepackage{listings}
\usepackage{longtable}
\usepackage{float}
\lstset{language=[90]Fortran,
	basicstyle=\ttfamily,
	keywordstyle=\color{red},
	commentstyle=\color{green},
	morecomment=[l]{!\ }% Comment only with space after !
}
\hypersetup{
	colorlinks=true,
	linkcolor=blue,
	filecolor=magenta,      
	urlcolor=cyan,
}
\urlstyle{same}
\DeclareGraphicsExtensions{.pdf,.png,.jpg,.jpeg}

\graphicspath{{pictures/}}
\title{\textbf{Практическое занятие 2  \\ -- \\ 
		История и методология развития методов электромагнитных полей}}
\author{И.А.Юхновский}
\date{2021}

\begin{document}
	
	\maketitle
	\thispagestyle{empty}
	\section*{Аннотация}
	Практическое занятие по теме лекции 2.3 История и методология развития методов электромагнитных полей
	
	\tableofcontents{}
	
	\section{Введение}
 	Цитата  ~\cite{kataev}.
	
	Рисунок
	\begin{figure}[H]
		\centering
		\includegraphics[width=\textwidth]{ris_1}
		\caption{Пациент  П., 44 года. Обширные оростомы в околоушно-жевательных областях, парез лицевого нерва слева ~\cite{rsj}.}
		\label{fig:ris_1}
	\end{figure}
	
	\section{Выводы}
	Вывод
	
	\section{Заключение}
	Заключение
	
	\begin{thebibliography}{3}
		\bibitem{kataev} Катаев С.С., Зеленина Н.Б., Шилова Е.А. Определение	дезоморфина в моче. Проблемы экспертизы в медицине.	2007; 1: 32-36 
		
	\end{thebibliography}
	
\end{document}	